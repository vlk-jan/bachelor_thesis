\section{Auxiliary functions}
    In this section, we will present the auxiliary functions used by the nodes of our BT algorithm.\\
    These will include the functions used for conversions, more complex or repetitive mathematical operations, and other functions that are not directly related to the BT algorithm.\\
    One of the big sections will be the algorithm used for determining the classification and cost of road segments in the road network.\\
    We will split the functions into categories based on their purpose.

    \subsection{Road cost algorithm}
        \label{sec:road_cost}
        We will use the algorithm developed during the summer of 2022 for the RobInGas project at the CTU CRAS. The algorithm was designed to determine the cost of crossing the road based on the road classification, curvature, and other factors.\\
        We will briefly present the functionality of the algorithm. The full description with implementation details can be found in \cite{Road_cost_docs}.\\
        This is also the part of the final solution in our thesis written in Python instead of \CC\ this is due to it being part of a different project. Other reasons include the usage of Python-specific libraries. While we could rewrite the code to \CC, it was not deemed necessary as this part is run only once at the beginning of the mission and therefore does not need to be optimized for speed.\\\\
        \bfc{Overview}\\
            We use multiple parameters to determine the cost of crossing. The most important ones are the geometrical properties of the road. This includes the curvature of the road, the elevation profile, and the proximity to intersections. We also use the road classification to add to the cost function.\\
            Other parameters would be beneficiary, such as the road width, the presence of a pedestrian crossing, and the expected traffic speed.\\
            Unfortunately, we do not have access to all of these parameters. We use the OSM data, which does not necessarily contain all those additional parameters. Therefore, we will inject this contextual information directly into the algorithm and deal with these parameters separately. The contextual information is discussed in section \ref{sec:context}.\\
            The OSM data also do not contain the elevation profile of the road. Acquiring this data is not straightforward, as it is not readily available through free or open-source channels. The elevation data we use were purchased from the Land Survey Office of the Czech Republic. We use the ZABAGED \cite{ZABAGED} data. This data from the Land Survey Office are available only for the area of the Czech Republic. However, any file with elevation data with the correct formatting can be used. The file format in use conforms to the following specifications: each line of the text file contains the easting, northing, and altitude coordinates for a single point, separated by a space. Each point is described in a separate line, and lines are separated using the newline character \texttt{"\textbackslash n"}.\\
            If the elevation data are not provided, the algorithm will still function. It will just not take the elevation profile into account. The road cost will be determined only from the curvate, proximity to intersections, and road classification.\\\\
        \bfc{Algorithm}\\
            The algorithm is divided into several parts.\\
            The first part is obtaining the road segments from downloaded OSM data. This part is also responsible for logging the road classification for each segment. The road segments from OSM data are divided into multiple smaller equidistant segments.\\
            The second part is responsible for determining the curvature of the roads. This is done by calculating the radius of the circumcircle of the triangle formed by the two adjacent road segments. This approach is visualized in image \ref{fig:curvature}. We then sort the road segments into multiple classes based on their radius. In this part, we also detect intersections and penalize the road segments close to them.\\
            In the third part, we determine the elevation profile of the road. We then classify the road segments using the TPI (Terrain Profile Index) method. Some TPI classes are presented in image \ref{fig:TPI}.\\
            In the last part, we combine the results from the previous parts and calculate the final cost of crossing for each segment. These costs are then saved to a file to be used later.\\\\
            \begin{figure}[ht]
                \centering
                \begin{subfigure}[b]{0.9\textwidth}
                    \centering
                    \includegraphics[trim={0cm 3.55cm 0cm 2.8cm}, clip, width=0.95\textwidth]{images/path_curvature.pdf}
                    \caption{Visualization of the circumcircle radii.}
                    \label{fig:curvature}
                \end{subfigure}
                \begin{subfigure}[b]{0.9\textwidth}
                    \centering
                    \includegraphics[trim={0cm 5.2cm 0cm 5.2cm}, clip, width=0.95\textwidth]{images/TPI_classification.pdf}
                    \caption{Representation of certain TPI classes.}
                    \label{fig:TPI}
                \end{subfigure}
                \caption{Visualization of key elements in the road cost algorithm.}
            \end{figure}
        \bfc{Usage}\\
            As was stated earlier, this algorithm is executed only once preferably during the pre-mission planning. Later we only keep the final costs, and based on them, we determine if the location where the robot is trying to cross is suitable and safe.\\
            We rely on ROS to enable the communication between our main algorithm and the algorithm for determining the suitability of the location for crossing. We implemented a ROS service for this purpose.\\
            Another part of the algorithm that could be implemented in future work is to try and provide the robot with a more suitable crossing place. This could be used if the pre-mission planning was not performed or the robot's path changed, and the current location is unsuitable. If such a place is found and provided, a cost map that will be used to change the current cost map of the path planner should be published. This is done so that the robot's controls are not overridden until we begin the crossing itself.\\
            The effect of this algorithm on a path planner is shown in image \ref{fig:path}.
            \begin{figure}[ht]
                \centering
                \includegraphics[width=0.53\textwidth]{images/path.png}
                \caption{The effect of the road cost algorithm on the path planning.}
                \label{fig:path}
            \end{figure}

    \subsection{Mathematical functions}
    \label{sec:math_func}
        \bfc{Difference between two angles}\\
            This function is used to determine the difference between two given angles. We assume the angles given are in radians, and both are in the interval $\langle0;2\pi\rangle$. This difference is calculated to be the smallest possible and to fit within the interval $\langle-\pi;\pi\rangle$. The formula we use is a modified version of the one from \cite{calc_rotation} and has the following form
            \begin{equation}
                \Delta\varphi = \left((\varphi_{2} - \varphi_{1} + \pi) \mod (2\pi)\right) - \pi.
            \end{equation}
            Before returning the result, we check whether the result is within the specified interval.

    \subsection{Geographical functions}
    \label{sec:geo_func}
        Here we will present the functions used for geographical calculations. These include conversions between coordinate systems, calculating azimuths, and others.\\\\
        \bfc{Converting GPS to UTM}\\
            While most geographical data are stored in the WGS84 coordinate system, generally known as GPS, the UTM coordinate system is more suitable for calculations. Therefore, we will convert the GPS coordinates to UTM.\\
            When working with geographical conversions in \CC, we use the library GeographicLib. The function from this library that provides the conversion is \texttt{GeographicLib::UTMUPS::Forward}. This function takes the point's latitude and longitude and returns the point's easting and northing in the UTM coordinate system. It also returns the zone number and whether the point is in the northern or southern hemisphere.\\
            We use the \texttt{utm} library when converting geographical data in Python. The function facilitating the conversion from WGS84 to UTM is \texttt{utm.from\_latlon}.\\\\
        \bfc{Converting NED to ENU}\\
            There are two possible orientations of an azimuth. The NED (North-East-Down) and the ENU (East-North-Up).\\
            NED means that azimuth 0 points north, and its value increases clockwise. This orientation is mainly used in cartography and everyday life.\\
            ENU means that azimuth 0 points east, and its value increases counterclockwise. This orientation is mainly used in navigation and robotics, as it is consistent with REP-103 \cite{REP-103}.\\
            In the entire project, we use the ENU orientation. However, as we rely on other ROS nodes to provide us with the azimuth, we need to be able to convert the azimuth from NED to ENU.\\
            The conversion should be much simpler since we do not deal with coordinates but with already computed azimuths.\\
            The image \ref{fig:dir_indi} shows the two possible orientations of the azimuth. This image also provides us with the insight we need to determine the conversion formula. We need to divide the formula into two parts.\\
            \begin{figure}[ht]
                \centering
                \includegraphics[width=0.5\textwidth]{images/direction_indicator.pdf}
                \caption{Two possible orientations of the azimuth.}
                \label{fig:dir_indi}
            \end{figure}
            \noindent The first option is when the azimuth (in NED) is between $0\:\si{\radian}$ and $\frac{\pi}{2}\:\si{\radian}$. In this case, the azimuth in ENU is computed in the following way
            \begin{equation}
                \varphi_{\rm{ENU}} = \frac{\pi}{2} - \varphi_{\rm{NED}},
            \end{equation}
            where $\varphi_{\rm{ENU}}$ is the azimuth in ENU and $\varphi_{\rm{NED}}$ is the azimuth in NED.\\
            The second option is for all other azimuths, e.g., when the azimuth is between $\frac{\pi}{2}\:\si{\radian}$ and $2\pi\:\si{\radian}$. In this case, the azimuth in ENU is computed in the following way
            \begin{equation}
                \varphi_{\rm{ENU}} = \frac{5\pi}{2} - \varphi_{\rm{NED}}.
            \end{equation}\\
        \bfc{Compute azimuth from coordinates}\\
            This function is used to compute the azimuth (in NED) for an observer standing at the first point and looking at the second point.\\
            We will use a slightly modified version of the formula from \cite{calc_bearing}. This calculation was created for the WGS84 coordinate system, however, we use the UTM coordinate system. Having said that, we can use the same formula, as the WGS84 to UTM projection is conformal \cite{Map_projections}. In testing the difference between calculated azimuths using the WGS84 and UTM coordinates was around $\Delta\varphi=0.097\:\si{\radian}$ or $\Delta\varphi=5.541\si{\degree}$.\\
            The computational equations are as follows
            \begin{align}
                \Delta y &= y_{2} - y_{1}, \\
                \alpha &= \sin{(\Delta y)}\,\cos{(x_{2})}, \\
                \beta &= \cos{(x_{1})}\,\sin{(x_{2})} - \sin{(x_{1})}\,\cos{(x_{2})}\,\cos{(\Delta y)}, \\
                \varphi &= \arctan{\left(\frac{\alpha}{\beta}\right)}, \\
                \varphi &= (\varphi + 2\pi) \mod (2\pi).
            \end{align}
            Where $x_{1}$ and $y_{1}$ are the latitude and longitude of the first point, and $x_{2}$ and $y_{2}$ are the latitude and longitude of the second point.\\\\
        \bfc{Compute heading for robot}\\
            The use of this function is to determine the heading of the robot. It is used to get the robot perpendicular to the road.\\
            The function takes the robot's azimuth and the road's heading. The algorithm creates two new variables, one $+\frac{\pi}{2}$ and one $-\frac{\pi}{2}$ from the road's heading. This is necessary as we do not know in what order road points are stored, and we do not need to differentiate the side we approach the road from.\\
            Then it computes the difference between the robot's heading and the two new azimuths. The smaller difference is then returned.
    
    \subsection{Contextual information and score}
    \label{sec:context}
        \bfc{Contextual information}\\
            Contextual information provides us with valuable information about the environment. This information is a vital part of choosing the best location for crossing.\\
            The contextual information we use is the following
            \begin{itemize}
                \item \textbf{Maximal speed} -- The maximal speed of vehicles on the road.
                \item \textbf{Number of lanes} -- The number of lanes on the road.
                \item \textbf{Road width} -- The width of the road.
                \item \textbf{Road type} -- The type of the road.
                \item \textbf{Pedestrian crossing} -- Whether there is a pedestrian crossing on the road and its location.
            \end{itemize}
        \bfc{Obtaining contextual information}\\
            Contextual information may be obtained from several sources. One source could be the OSM database. However, this database is not always up to date, and there is no guarantee that the necessary information will be available. Other sources could be the direct observations of the environment or the information provided by other ROS nodes.\\
            In our case, the operator will submit the information. We have prepared a Python class with the appropriate variables and functions. To use the contextual information, the operator should create an instance of this class and fill in the appropriate variables.\\
            The creation of the class is recommended to be done in advance. It can be automated or done manually. The class also contains functions necessary for saving and loading contextual information to a file.\\
            During the execution of our algorithm, the tree node will call a ROS service to obtain the contextual information. This service will return the contextual information for the closest road to the requested location.\\
            By employing this approach, we can effectively capture and record contextual information for multiple roads, which in turn enables us to perform multiple road crossings during a mission.\\\\
        \bfc{Calculating the context score}\\
            The final score is determined by the sum of points assigned to each contextual information.
            \begin{equation}
                \xi_{\rm{context}} = \sum_{i=1}^{n} \xi_{\rm{context},i},
            \end{equation}
            since we currently have only five distinct types of contextual information, we set $n$ equal to 5.\\
            The individual points are set as follows
            \begin{itemize}
                \item \textbf{Maximal speed} -- $v_{\rm{max}}$
                    \begin{itemize}
                        \item $v_{\rm{max}}\leq30\quad\to\quad\xi_{\rm{context,1}}=3$
                        \item $v_{\rm{max}}\leq50\quad\to\quad\xi_{\rm{context,1}}=2$
                        \item $v_{\rm{max}}\leq80\quad\to\quad\xi_{\rm{context,1}}=1$
                    \end{itemize}
                \item \textbf{Number of lanes} -- $n_{\rm{lanes}}$
                    \begin{itemize}
                        \item $n_{\rm{lanes}}=1\quad\to\quad\xi_{\rm{context,2}}=5$
                        \item $n_{\rm{lanes}}=2\quad\to\quad\xi_{\rm{context,2}}=4$
                        \item $n_{\rm{lanes}}=3\quad\to\quad\xi_{\rm{context,2}}=2$
                        \item $n_{\rm{lanes}}=4\quad\to\quad\xi_{\rm{context,2}}=1$
                    \end{itemize}
                \item \textbf{Road width} -- $w_{\rm{road}}$
                    \begin{itemize}
                        \item $w_{\rm{road}}\leq3.5\quad\to\quad\xi_{\rm{context,3}}=4$
                        \item $w_{\rm{road}}\leq4.5\quad\to\quad\xi_{\rm{context,3}}=3$
                        \item $w_{\rm{road}}\leq5.5\quad\to\quad\xi_{\rm{context,3}}=2$
                        \item $w_{\rm{road}}\leq6.5\quad\to\quad\xi_{\rm{context,3}}=1$
                    \end{itemize}
                \item \textbf{Road type}
                    \begin{itemize}
                        \item motorway $\quad\to\quad\xi_{\rm{context,4}}=-50$
                        \item trunk $\quad\qquad\to\quad\xi_{\rm{context,4}}=-50$
                        \item primary $\qquad\to\quad\xi_{\rm{context,4}}=1$
                        \item secondary $\,\quad\to\quad\xi_{\rm{context,4}}=2$
                        \item tertiary $\,\qquad\to\quad\xi_{\rm{context,4}}=3$
                    \end{itemize}
                \item \textbf{Pedestrian crossing}
                    \begin{itemize}
                        \item if present $\quad\to\quad\xi_{\rm{context,5}}=100$
                    \end{itemize}
            \end{itemize}
