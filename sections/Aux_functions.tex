\section{Auxiliary functions}
    In this section we will present the auxiliary function used in the nodes of our BT algorithm.\\
    These will include the functions used for conversions, more complex or reperirive mathematical operations, and other functions that are not directly related to the BT algorithm.\\
    One of the big sections will be the algorithms used for determining the classification and cost of roads in the road network.\\
    We will split the functions into categories based on their purpose.

    \subsection{Road cost algorithm}
        We will use the algorithm developed during the summer of 2022 for the RobInGas project at the CTU CRAS. The algorithm was designed to determine the cost of crossing the road based on the road classification, curvature, and other factors.\\
        We will briefly present the functionality of the algrithm. The full description with implementation details can be found in \cite{Road_cost_docs}.\\
        This is also the only part of our thesis that is written in Python instead of \CC\ this is due to it being part of different project. Other reasons include the usage of libraries for Python. While we could rewrite the code to \CC\ it was not deemed necessary as this part is run only once at the beginning of the mission and therefore it does not need to be optimized for speed.\\\\
        \bfc{Overview}\\
            We use multiple parameters in order to determine the cost of crossing. The most important ones are the geometrical properties of the road. This includes the curvature of the road, the elevation profile, and the proximity to intersections. We also use the road classification to add to the cost function.\\
            Other parameters would be beneficiary such as the road width, the presence of a pedestrian crossing, and the expected speed of the traffic.\\
            Unfortunately, we do not have access to all of these parameters. We use the OSM data which do not necessarily contain all of those additional parameters. We therefore will inject this information directly to the algorithm and deal with these parameters separately.\\
            The OSM data also do not contain the elevation profile of the road. This data is also not easily obtainable from free or open-source sources. The elevation data we use were purchased from the Land Survey Office of the Czech Republic. We use the ZABAGED format\footnote{\url{https://geoportal.cuzk.cz/(S(dcpfei0nmxcwgoe4frurwfgm))/Default.aspx?lng=EN&mode=TextMeta&text=dSady_zabaged&side=zabaged&menu=24}}. This data from the Land Survey Office are available only for the area of the Czech Republic. However any file with elevation data with the correct formatting can be used. The used file format is as follows. A text file with the easting, northing and altitude of one point on one line separated by a space. The lines are separated with a newline character \texttt{\textbackslash n}, and each line describes exactly one point.\\
            If the elevation data are not provided the algorithm will still function, it will just not take the elevation profile into account. Meaning the cost of the roads will be determined only from the curvate and road classification.\\\\
        \bfc{Algorithm}\\


    \subsection{Mathematical functions}

    \subsection{Geographical functions}
    \label{sec:geo_func}