In this thesis, our objective was to design, implement, and evaluate an algorithm for safely crossing public roads with a middle-sized mobile robot. After evaluating various development approaches, we used a behavior tree as the main control algorithm structure. This decision allowed us to create a modular and flexible algorithm that can be easily modified or extended.\\
The algorithm was designed to fulfill three main tasks: determining the suitable position for crossing, changing the location if necessary, and executing the crossing maneuver itself. The implementation involved utilizing the \CC\ and Python programming languages, with \CC\ handling the robot's maneuvering tasks and Python managing map-related tasks. It is worth mentioning that the navigation to a better position for crossing still needs to be implemented and remains open for future work.\\
To evaluate the algorithm's performance, we proposed a metric to determine its suitability and optimality. We conducted tests both in simulation and in a real-world environment away from public roads. Simulation tests took place in the Gazebo Classic simulator, while the real-world experiments were in a controlled environment under the supervisor's supervision. However, one limitation we encountered was the absence of a vehicle detection node, preventing us from testing the algorithm with actual vehicles.\\
The results of the experiments demonstrated that the algorithm successfully executed the crossing maneuver in a safe manner. Additionally, it displayed adaptability to changes in the environment and exhibited robustness in handling certain errors that may occur in the vehicle detection node.\\
Overall, the developed algorithm showcases promising potential for the safe crossing of public roads with a middle-sized mobile robot. Future work could focus on implementing the remaining task from the design stage and perform further testing of the algorithm with real vehicles to enhance its practicality and real-world applicability.\\
The work we performed aims to enable the use of mobile robots in real-world environments. We believe the results will contribute to the development of autonomous robots and their use in missions that are too dangerous or repetitive for humans. Before the deployment of such robots, further research and testing is required, as well as the need for a legal framework to regulate their use in public spaces.
