\section{Behavior tree nodes}
    Here we will present the methods used to implement the individual nodes of our BT algorithm. We will split the nodes into categories based on the sub-tree they belong to.
    \subsection{Introduction}
        The BT algorithm is implemented using the behaviortree-cpp-v3 library. We will therefore present the way we create, implement and run the tree. We will also show how nodes are created, implemented and used.\\\\
        \bfc{Creating the node}\\
            To create the node we need to create a class that inherits from one of the parent classes present in the library. The parent classes depend on the type of node we want to create. If we are creating an action node, we will inherit from the \texttt{SyncActionNode} class. If we are creating a condition node, we will inherit from the \texttt{ConditionNode} class.\\
            For all nodes we must define two functions the \texttt{tick} and \texttt{providedPorts} functions.\\
            The \texttt{tick} function is responsible for the actual implementation of the node. It is called every time the node is executed. It is also responsible for returning the state of the node.\\
            The \texttt{providedPorts} function is responsible for defining the ports the node will use. This function must be defined even if the node does not use any ports.\\\\
        \bfc{Using the node}\\
            To use the node, we first have to create a tree. We do this by creating a class \texttt{Road\_cross\_tree} in which we have a \texttt{BehaviorTreeFactory} object. This object is used to create the tree. We also have a \texttt{Tree} object which is used to store and run the tree.\\
            If we want to use a node, we have to register it in the \texttt{BehaviorTreeFactory} object. We do this by calling the \texttt{registerNodeType} function. For this function we need to specify the class of the node we want to register, eg. one of the nodes we created. It also takes one \texttt{std::string} parameter which identifies the name of the node in our BT algorithm \texttt{.xml} file.\\\\
        \bfc{Creating and running the tree}\\
            To create the tree, we first must have a \texttt{.xml} file with the tree structure. This file is then parsed by the \texttt{BehaviorTreeFactory} object using the \texttt{createTreeFromFile} function. The result is a \texttt{Tree} object which we can then run.\\
            To run the tree, we call the \texttt{tickRoot} function on the \texttt{Tree} object. This function returns the state of the root node of the tree.\\\\
        \bfc{Blackboard}\\
            The blackboard is a shared memory between the nodes of the tree. It is used to store data that is being used by multiple nodes.\\
            It is also the reason why we implement the \texttt{providedPorts} function. This function specifies the ports the node will use. These ports can take a constant value (specified during the creation of the tree structure) or look for a value in the blackboard.\\\\
        \bfc{Logging}\\
            The BT library also provides us with a logging functionality. This functionality is useful for debugging and testing.\\
            We can use different types of loggers. The most common one is the \texttt{StdCoutLogger} which logs to the standard output. We can also use the \texttt{FileLogger} which logs to a file.\\
            We will use the \texttt{FileLogger} to log the tree execution. This logger is useful for its integration with the \texttt{Groot} application as we can import the produced file and visualize the tree execution.\\
            The logging will be used only for debugging and testing purposes and serves no other function in the final product.
    \subsection{Init BT}
\label{sec:Init-BT-impl}
    Here we will present the nodes used in the Init sub-tree. This sub-tree is used to initialize the BT algorithm. It is the first sub-tree to be executed. This tree is going to be executed only once per road crossing.\\\\
    \bfc{GetPosition -- Action node}\\
        This node is responsible for obtaining the current GPS position of the robot and converting it to the UTM coordinate system. It is implemented as a ROS topic subscriber. The topic subscribed is \texttt{/gps/fix} where the GPS data are being published.\\
        The obtained data are then converted to UTM using the \texttt{gps\_to\_utm} function defined in \ref{sec:geo_func}. The result is then stored as two BT blackboard variables -- \texttt{easting} and \texttt{northing}.\\
        For every obtained value, it also calls a ROS service \texttt{place\_suitability} to determine the suitability of the current position for crossing.\\\\
    \bfc{CrossRoad -- Condition node}\\
        This node tells our algorithm if we are close enough to a road to take over the robot's controls. If we are not the path-planning or other node is left in control.\\
        We use the return values of the ROS service call issued in the \texttt{GetPosition} node. This service has two return values -- \texttt{validity} and \texttt{suitability}. \texttt{Suitability} uses the road cost as well as context score to judge the place for crossing. For \texttt{validity}, we only calculate the distance of the current location to road segments from OSM.\\
        Therefore the \texttt{validity} variable is the one determining the output of this node. The distance limit we proposed as sufficient is $10\:\rm{m}$ from the center of the road.\\\\
    \bfc{PlaceSuitable -- Condition node}\\
        This node states whether the current robot's location, stored as a blackboard variable, is suitable for crossing.\\
        It uses the second return value from the ROS service called in the \texttt{GetPosition} node. As stated, this value takes into account the road cost for our location from the road-cost algorithm (\ref{sec:road_cost}) and the context score calculated separately before the service call.\\
        The context score is based on the contextual information that is available to us. This information may be passed from other nodes (e.g., computer vision node for detecting road parameters) or set by the operator.\\
        The calculation of the context score and the process of obtaining the contextual information is described in \ref{sec:context}.\\\\

    \noindent Other nodes shown it the BT structure (fig \ref{fig:Init-BT}) are currently returning \texttt{FAILURE}. These nodes are there to show the potential for further work. Their main purpose is to steer the robot to a more optimal location for crossing. In this work, we assume that the correct location was chosen in the pre-mission planning.
    