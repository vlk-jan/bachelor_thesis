\section{Software}
    All programming work in this thesis is aimed to work with the Robot Operating System (ROS) \cite{ROS}. More specifically, we will use ROS1 in version Noetic Ninjemys\footnote{\url{http://wiki.ros.org/noetic}}.\\\\
    \bfc{Programming languages}\\
        The majority of implementation work will be done in a \CC\ programming language. The version of \CC\ standard used is \CC14, as it is the default for the ROS version we use.\\
        The \CC\ language was chosen for its speed and efficiency. It was also chosen for compatibility with some of the libraries we need to use for our project.\\
        The second programming language we will use is Python in version 3.8. Python was chosen for its simplicity and ease of use, as well as for integrating our previous work in OSM data processing.\\\\
    \bfc{BT library}\\
        There are a few possibilities regarding the BT library we can use for our solution. As BTs are not very commonly used, the choice is more limited than if we were to use an FSM. Another limiting factor we imposed is support or direct integration with ROS.\\
        We still have a few options, and we can even choose a programming language in which to implement the BT nodes. The two programming languages with the most library options are \CC\ and Python. This copies the ROS mentality, where these two languages are natively supported. Several available BT libraries are discussed here \cite{BT_FSM}.\\
        We have decided to use a \CC\ behaviortree-cpp-v3 library\footnote{\url{https://github.com/BehaviorTree/BehaviorTree.CPP}}. The choice was made for multiple reasons. This library was written with deployment in ROS in mind. Moreover, it is regularly updated and maintained, making it a safe choice for us. It also comes with documentation that will be helpful during the implementation process. There are two versions of the documentation \cite{BT_docs} and \cite{BT_docs_new}. We will mainly use the newer one (the second mentioned), but we will cross-reference it with the older one.\\
        Another benefit of this implementation is that it comes with a GUI application for creating BTs called Groot\footnote{\url{https://github.com/BehaviorTree/Groot}}. This application creates an \texttt{.xml} file with the BT structure we can import into our code later.\\\\
    \bfc{Libraries for OSM and work with geographical data}\\
        There are a lot of libraries to choose from when it comes to working with OSM data. These libraries are created for different programming languages and have different features.\\
        Even though the majority of our work was written in \CC, we were building on top of previous work of assigning costs to road segments in OSM data. This work was done during the 2022 summer as a part of the RobInGas project here at the CTU under the Center for Robotics and Autonomous Systems (CRAS\footnote{\url{https://robotics.fel.cvut.cz/cras}}) group.\\
        The work was done in Python, and the library used was the overpy library\footnote{\url{https://github.com/DinoTools/python-overpy}}. This library is used to access the OSM Overpass API and download the map data. The Overpass API (formerly known as OSM Server Side Scripting) is a read-only API that serves up custom-selected parts of the OSM map data. The difference between the main API is that the Overpass API is optimized for small to large consumers (up to roughly 10 million elements). Many services and applications use it as a database backend.\cite{Overpass}\\
        Other libraries used for work with the OSM data were shapely\cite{shapely} and numpy\cite{numpy}. These libraries were used to classify and assign costs to individual road segments in the downloaded OSM data.\\
        In our work, we also need to convert the coordinates of the robot from the GPS coordinate system to the UTM coordinate system. The conversion is done using the GeographicLib library\cite{GeographicLib} in \CC\ and utm\footnote{\url{https://github.com/Turbo87/utm}} in Python.