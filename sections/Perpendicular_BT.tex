\subsection{Perpendicular BT}
\label{sec:Perpendicular-BT-impl}
    The task of this tree is to position the robot perpendicular to the road. It is the second sub-tree to be executed, and as well as Init BT, it is used to prepare the robot for the crossing and is only executed once.\\\\
    \bfc{GetAzimuth -- action node}\\
        This node is responsible for obtaining the robot's current azimuth. We have a ROS subscriber listening to topic published by \texttt{compass} node\footnote{\url{https://github.com/ctu-vras/compass}}.\\
        The compass node may publish the azimuth in several different formats. In our program, we use the ENU format in radians. But if the compass node publishes the azimuth in a different format, we have subscribers that can convert it to the desired format.\\
        The azimuth is then stored as a blackboard variable \texttt{azimuth}.\\\\
    \bfc{RoadHeading -- action node}\\
        This node calculates the heading of the closest road to the robot. We take the current robot's position from the blackboard variable \texttt{easting} and \texttt{northing} and send a request to the ROS service \texttt{get\_road\_heading}.\\
        The service returns the two coordinate points representing the closest road segment's starting and ending points.\\
        We then calculate the heading of the road segment using the function defined in section \ref{sec:geo_func}.\\
        The calculated road heading is then stored as a blackboard variable \texttt{road\_heading}.\\\\
    \bfc{ComputeHeading -- action node}\\
        This node uses the blackboard variable \texttt{azimuth} and \texttt{road\_heading} to calculate the heading the robot should achieve to be perpendicular to the road.\\
        The calculation is defined in section \ref{sec:geo_func}.\\
        The result is stored as a blackboard variable \texttt{req\_azimuth}.\\\\
    \bfc{RobotPerpendicular -- condition node}\\
        This node checks if the robot is perpendicular to the road. It works by comparing the current azimuth with the required heading. Both of these values are stored in the blackboard.\\
        We use the function defined in section \ref{sec:math_func} to compare the values to calculate the difference between two angles. The result is then compared to the threshold value, which is set to $0.1745\:\rm{rad}$ or $10\rm{^{\circ}}$.\\\\
    \bfc{RotateRobot -- action node}\\
        This node rotates the robot to the required heading.\\
        First, we calculate the difference between the robot's current and desired azimuth. Then, based on the difference, we proportionally set the rotation direction and speed.\\
        The calculated movement is then published to the \texttt{cmd\_vel} topic.\\\\
    \bfc{StepFromRoad -- action node}\\
        If, for whatever reason, the robot is not able to rotate safely, primarily due to the possibility of ending on the road, we use this node to move the robot away from the road.\\
        Firstly we check the difference between the robot's current azimuth and the road heading. Based on the difference, we set the direction of the movement.\\
        The movement is then published to the \texttt{cmd\_vel} topic.
