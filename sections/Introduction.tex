\chapter*{Introduction}
    In today's world, mobile robots are increasingly being utilized in a variety of applications. In many of these applications, the robots must cross roads to achieve their goals, making it essential to design an algorithm that enables the robot to cross the road safely.\\
    The algorithm should be able to determine whether the current place is suitable for crossing. The algorithm will accept contextual inputs such as expected vehicle velocity, road type, number of lanes, and the presence of a pedestrian crossing with or without traffic lights. These data will be used to assess the current situation and determine whether it is safe to cross the road. If it is, it should facilitate the crossing itself. If the location is not suitable, the algorithm should provide a reason and suggest a more appropriate location nearby. The algorithm must also be designed to operate on different robots with various sensor configurations and on all roads without any additional limitations.\\
    This thesis aims to provide a theoretical background to the problem and explore possible solutions. We will also present the hardware and software used for real-world and simulation experiments. We will discuss our chosen approach and its functionality and present the algorithms we developed and implemented. Finally, we will explain the results of our experiments and discuss their significance.\\
    Our work also depends on the output of other projects, such as vehicle detection and localization or path planning. We cannot rely on these projects to be completed or entirely functional. Therefore, we need a way to simulate and test our algorithm without them.\\
    In simulation experiments, we will inject data directly into our algorithm. For real-world experiments, we will try to use the outcomes of the projects mentioned earlier. However, we can inject data directly into our algorithm, provided the projects are not finished or functional.\\
    \newpage
    \section*{Used abbreviations}
        \begin{itemize}
            \item \textbf{AI} -- Artificial Intelligence
            \item \textbf{API} -- Application Programming Interface
            \item \textbf{BT} -- Behavior Tree
            \item \textbf{CRAS} -- Center for Robotics and Autonomous Systems
            \item \textbf{ENU} -- East-North-Up
            \item \textbf{FSM} -- Finite State Machine
            \item \textbf{GNSS} -- Global Navigation Satellite System
            \item \textbf{GPS} -- Global Positioning System
            \item \textbf{GUI} -- Graphical User Interface
            \item \textbf{HFSM} -- Hierarchical Finite State Machine
            \item \textbf{IMU} -- Inertial Measurement Unit
            \item \textbf{LiDAR} -- Laser imagining Detection and Ranging
            \item \textbf{NED} -- North-East-Down
            \item \textbf{NPC} -- Non-Player Character
            \item \textbf{OSM} -- Open Street Map
            \item \textbf{REP} -- ROS Enhancement Proposals
            \item \textbf{RL} -- Reinforcement Learning
            \item \textbf{ROS} -- Robot Operating System
            \item \textbf{TPI} -- Terrain Profile Index
            \item \textbf{UTM} -- Universal Transverse Mercator
            \item \textbf{WGS84} -- World Geodetic System 1984
            \item \textbf{ZABAGED} -- Základní báze geografických dat (Basic database of geographic data)
        \end{itemize}
    