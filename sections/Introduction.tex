\chapter*{Introduction}
    In today's world, mobile robots are increasingly being utilized in a variety of applications. In many of these applications, the robots must cross roads to achieve their goals, making it essential to design an algorithm that enables the robot to cross the road safely.\\
    The algorithm should be able to determine whether the current place is suitable for crossing. The algorithm will accept contextual inputs such as vehicle velocity data, road type, number of lanes, and the presence of a pedestrian crossing with or without traffic lights. These data will be used to assess the current situation and determine whether it is safe to cross the road. If it is, it should facilitate the crossing itself. If the location is not suitable, the algorithm should provide a reason and suggest a more appropriate location nearby. The algorithm must also be designed to operate on different robots with various sensor configurations and on different road types without any limitations.\\
    This thesis aims to provide a theoretical background to the problem and explore possible solutions. We will also present the hardware and software used for real-world and simulation experiments. We will discuss our chosen approach and its functionality and present the algorithms we developed and implemented. Finally, we will explain the results of our experiments and discuss their significance.\\
    Our work also depends on the output of other projects, such as vehicle detection and localization or path planning. We cannot rely on these projects to be completed or entirely functional. Therefore, we need a way to simulate and test our algorithm without them.\\
    In simulation experiments, we will inject data directly into our algorithm. For real-world experiments, we will use the outcomes of the respective projects. However, we can inject data directly into our algorithm, provided the projects are not finished or functional.
    %In real-world applications, mobile robots are often required to cross roads to achieve their tasks. Therefore, it is necessary to design an algorithm that will allow the robot to cross the road safely. The algorithm should be able to determine if the current place is suitable for crossing, and if it is, it should perform the crossing itself. If it is not, it should provide a reason for not being so and potentially suggest a more suitable place in its vicinity.\\
    %Moreover, the algorithm should not be limited to a specific robot or a specific road. It should be capable of operating on a wide variety of robots with different sensor configurations as well as on different road types.\\
    %We will provide a brief theoretical background of the problem and show possible solutions. We will also present the hardware and software used for real-world and simulation experiments.\\
    %We will discuss the chosen approach and show its functionality. We will present developed algorithms and their implementations. Finally, we will explain the results of the experiments and discuss their significance.\\
    %In our task, we will use the ROS framework, a collection of tools, libraries and conventions for robot software development.\\
    %Our mission also depends on the output of other projects, such as vehicle detection and localization or path-planning node. For simulation purposes, we will be injecting data directly into our algorithm. As for the real-world experiments, we will try to use the developed nodes, but we will also be able to inject data directly into our algorithm.
    \newpage
    \section*{Used abreviations}
        \begin{itemize}
            \item \textbf{AI} -- Artificial Intelligence
            \item \textbf{API} -- Application Programming Interface
            \item \textbf{BT} -- Behavior Tree
            \item \textbf{CRAS} -- Center for Robotics and Autonomous Systems
            \item \textbf{ENU} -- East-North-Up
            \item \textbf{FSM} -- Finite State Machine
            \item \textbf{GPS} -- Global Positioning System
            \item \textbf{GUI} -- Graphical User Interface
            \item \textbf{HFMS} -- Hierarchical Finite State Machine
            \item \textbf{LiDAR} -- Laser imagining Detection and Ranging
            \item \textbf{NED} -- North-East-Down
            \item \textbf{OSM} -- Open Street Map
            \item \textbf{REP} -- ROS Enhancement Proposals
            \item \textbf{RL} -- Reinforcement Learning
            \item \textbf{ROS} -- Robot Operating System
            \item \textbf{TPI} -- Terrain Profile Index
            \item \textbf{UTM} -- Universal Transverse Mercator
            \item \textbf{ZABAGED} -- Základní báze geografických dat (Basic database of geographic data)
        \end{itemize}
    