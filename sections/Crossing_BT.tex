\subsection{Crossing BT}
\label{sec:Crossing-BT-impl}
    This tree is used for the main decision-making process. It is the third sub-tree to be executed, and the only one to be executed repeatedly.\\
    In multiple nodes we will use data about the detected vehicles, and collision parameters for each vehicle. Therefore, we first need to define the data structures used to store this information.\\\\
    \bfc{Vehicle data}\\
        The data structure for storing the information about the detected vehicles is defined as follows:
        \begin{lstlisting}[language=C++, caption={Vehicle data structure}, label={lst:vehicle_data}]
            struct vehicle_info {
                int id;
                double x_pos;
                double y_pos;
                double x_dot;
                double y_dot;
                double x_ddot;
                double y_ddot;
                double length;
                double width;
            };
            struct vehicles_data {
                int num_vehicles;
                std::vector<vehicle_info> data;
            };
        \end{lstlisting}
        The first struct \texttt{vehicle\_info} is used to store the information about single detected vehicle. The positions of the vehicles are expressed with relation to the robot frame, meaning the center of the robot is the origin of the coordinate system.\\
        The second struct \texttt{vehicles\_data} is used to store the information about all detected vehicles.\\\\
    \bfc{Collision data}\\
        The data structure for storing the collision parameters is defined as follows:
        \begin{lstlisting}[language=C++, caption={Collision data structure}, label={lst:collision_data}]
            struct collision_info {
                int car_id;
                double v_front;
                double v_back;
                bool collide;
            };
            struct collisions_data {
                int num_collisions;
                std::vector<collision_info> data;
            };
        \end{lstlisting}
        The first struct \texttt{collision\_data} is used to store the collision parameters for single vehicle.\\
        The \texttt{v\_front} and \texttt{v\_back} variables are the velocities of the robot to come into contact with the front or back of the vehicle. The figure \ref{fig:collision} shows the contact points we are calculating the velocities for.\\
        The \texttt{collide} variable is a boolean value that tells us if the robot is going to collide with the vehicle. It is calculated based on the current velocities of the robot and the vehicle.\\
        The second struct \texttt{collisions\_data} is used to store the information about all collisions.\\\\
    \bfc{Calculating the collision parameters}\\
        First we need to state the assumptions we are making in order to simplify the calculation.\\
        The first assumption is about the coordinate system we are using. We are using the robot frame, where the center of the robot is the origin of the system and all the positions are expressed with relation to this origin. The $x$-axis is pointing forward and the $y$-axis is pointing to the left. We can make this assumption because the calculations are done periodicaly and the results are only relevant for the current time step. It also simplifies the process as the vehhicle positions are already expressed in the robot frame.\\
        The second assumption is about the movement of the robot. We assume that the robot is moving in a straight line with constant velocity. This is reasonable as we want the robot to be as much predictable as possible, so we do not want to move the robot to the side. Assumption about the constant velocity, meaning the acceleration is zero is also reasonable. The robot's speeds are much smaller than the robot's acceleration, so we can neglect the acceleration.\\
        The third assumption is that we will calculate the collision only in two dimensions. This is reasonable as we are only interested in the information if the collision will occur or not. The area over which the collision can occur is relatively small and therefore any terrain deviation will not have a significant impact on the collision.\\
        The figure \ref{fig:collision} depicts a schematic view of the collision. The robot is shown in yellow and the vehicle in gray. There are shown two contact points. The first one (blue) is the point where the robot is going to collide with the front of the vehicle. The second one (red) is the point where the robot is going to collide with the back of the vehicle.\\
        \begin{figure}[ht]
            \centering
            \includegraphics[height=6cm]{images/collision.pdf}
            \caption{Visualization of collision points, coordinate systems, and vehicle parameters.}
            \label{fig:collision}
        \end{figure}
        \noindent For the first point we calculate the velocity \texttt{v\_front}. This velocity depicts the minimal speed of the robot to cross in front of the vehicle. For the second point we calculate the velocity \texttt{v\_back}. This velocity depicts the maximal speed of the robot to cross behind the vehicle.\\
