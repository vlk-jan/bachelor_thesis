\section{Maps}
    We will use the maps from the OpenStreetMap (OSM) project\footnote{\url{https://www.openstreetmap.org}}. The maps will be used to determine the surroundings of the robot and whether the current position is suitable for crossing.\\
    OSM is a project that creates and distributes free geographic data. The data is created by the community of users and is available for anyone to use. \cite{OSMwiki}\\
    The map data are expressed by a node, a way, or a relation. A node is a singular point in a map, it could be a landmark, a corner of a building, or a spot on the road. A way is an object created from multiple nodes. It can be either closed or open. Closed ways may represent a park, building, or other type of area. Open ways commonly represent roads, rivers, or other linear features. The relation is a collection of nodes, ways, or other relations. It is used to describe more complex objects, such as a bus line, a building complex, etc.\\
    We must also clarify the terminology we will use regarding azimuth and heading.\\
    \emph{An azimuth is a bearing, more precisely, a compass bearing from a specific point of observation like a radar station. A heading (in the general case of moving "forward") is the direction your nose is pointed in.}\\
    This description is taken from \cite{heading}.
