\section{Mathematical aparatus}
    This thesis will use chapters from linear algebra, calculus, geometry, and probability theory.\\
    The used theory will be shown and briefly explained in sections where it is needed. We will not provide precise definitions or state used theorems as it is outside the scope of this work. However, a book, a paper, or an online document with the corresponding information will be provided should the reader require a more thorough explanation.\\

\section{Maps}
    We will use the maps from the OpenStreetMap (OSM) project\footnote{\url{https://www.openstreetmap.org}}. The maps will be used to determine the surroundings of the robot and whether the current position is suitable for crossing.\\
    OSM is a project that creates and distributes free geographic data. The data is created by the community of users and is available for anyone to use.\cite{OSMwiki}\\
    The map data are expressed either by a node, a way, or a relation. Node is a singular point in map, it could be a landmark, a corner of building or a spot on the road. Way is an object created from multiple nodes. It can be either closed or open. Closed ways may represent a park, building or a some other type of area. Open ways commonly represent roads, rivers, or other linear features. Relation is a collection of nodes, ways, or other relations. It is used to describe more complex objects, such as a bus line, a building complex etc.