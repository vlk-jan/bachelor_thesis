The challenge of robot navigation in urban environments is a widely acknowledged and extensively studied problem within the field of autonomous robotics. However, the specific challenge of autonomous road crossing for robots is more specialized, resulting in a comparatively limited body of research in this domain. In this chapter, we will explore the existing literature and relevant works in the field of road crossing by autonomous agents.\\
Most of the research in the field of autonomous crossing concentrates on the challenge of crossing the road at a pedestrian crossing. These studies encompass both pedestrian crossings with and without traffic lights.\\
Chand and Yuta proposed a solution for navigation and path planning of road crossing for robots in urban areas \cite{Chand}. The proposed navigation strategy was separated into four phases. The first phase being sidewalk detection and navigation. In this phase, the robot moved along the sidewalk and navigated past obstacles. The second phase was navigation toward the push button. As these buttons are only found at crossings, it was used to navigate the robot to it. The third phase was approaching the crossing. The use of previously recorded position data was used to determine the position of the crossing. The fourth and final phase was the crossing itself. The robot used a camera to detect the traffic lights and determine the path during the crossing maneuver.\\
The work by Radwan et al. relies on laser and radar data to teach a Random Forest classifier to predict the safety of crossing the road \cite{Radwan}. During the evaluation of the crossing's safety, the robot gathered data from sensors over a short time interval. It then used the learned classifier to determine whether it was safe to cross the road. While this study provides a complete, robust, and reliable safety prediction, the result is a binary decision. No control algorithm was proposed or designed to facilitate the movement.\\
In \cite{Baker}, Baker and Yanco utilize a vison-based system to determine the traffic situation. Their solution uses two cameras with a view to the sides to detect incoming traffic and determine the safety of the crossing. The tracking of traffic situations is also performed during the crossing maneuver. Since this work was carried out for assistive robots, which could be installed in wheelchairs, it was designed assuming the crossing would occur at a designated pedestrian crossing. This paper only presents an algorithm for vehicle tracking without the implementation of a control algorithm. Moreover, the tracking algorithm assumes the velocity of the vehicles is constant.\\\\
There are also several works regarding the safe crossing of an intersection by autonomous vehicles. While these works are not directly applicable to the problem of crossing the road by a robot, they do provide some insight into the issue of autonomous maneuvering in an urban environment regarding the determination of the safety of the crossing.\\
The use of communication between vehicles, traffic structures, and traffic lights could be used to enable and improve autonomous operation. This approach, with its possibilities and drawbacks, was explored by Torres and Malikopoulos in \cite{CAV}. Moreover, the communication between vehicles and traffic infrastructure could be utilized by robots to allocate the right of way for the crossing. Autonomous cooperative driving was also explored by Lee and Park in \cite{Lee} and Campos et al. in \cite{Campos}.\\
Approaches to autonomous crossing of intersections without vehicle-to-vehicle communication were also explored. In \cite{CMMPHD}, the authors present a general-purpose multi-sensor tracking algorithm using a classifying multiple-model PHD filter.\\\\
In contrast to the aforementioned works, which primarily focused on sensor selection and data processing techniques or were capable of crossing only if specific requirements were met, our primary focus was the development of universal control algorithm facilitating the crossing maneuver.\\
The designed control algorithm is universal in the sense that it does not require specific sensors. It can utilize any sensor or combination of sensors. The only requirement is that the output of the data processing is in the form of vehicle data, i.e., position, velocity, and acceleration. The results of different crossing safety prediction algorithms can also be used as input.\\
The algorithm is also universal in the sense that it can be used for crossing at any location. It is, therefore, not only limited to designated pedestrian crossings. This allows for autonomous robotic applications in a broader range of urban environments.\\\\
The problem of autonomous road crossing is complex, requiring not only sufficient sensor data and processing but also a robust and reliable control algorithm. Another important aspect is the safety of the robot and its surroundings. Therefore, it is vital to ensure that all necessary safety measures are in place.\\
Henceforth, there is a need for a legal framework to regulate the operation of autonomous agents in urban environments and public spaces. This framework should ensure the safety of the agent and its surroundings while also allowing for the development of new technologies. The responsibilities of the agent and the human operator should be clearly defined.\\
Until such a framework is in place, the deployment of autonomous agents will be limited to controlled environments, such as factories and warehouses.
