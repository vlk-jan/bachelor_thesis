The challenge of robot navigation in urban environments is a widely acknowledged and extensively studied problem within the field of autonomous robotics. However, the specific challenge of autonomous road crossing for robots is more specialized, resulting in a comparatively limited body of research in this domain. In this chapter, we will explore the existing literature and relevant works in the field of road crossing by autonomous agents.\\
Most of the research in the field of autonomous crossing concentrates on the challenge of crossing the road at a pedestrian crossing. These studies encompass both pedestrian crossings with and without traffic lights.\\
Bauer et al. have presented a pedestrian robot with autonomous navigation in urban areas \cite{Bauer}. This robot can traverse the road by detecting and classifying traffic lights. This robot's crossing abilities are, however, limited only to signaled pedestrian crossings.\\
The work by Radwan et al. relies on laser and radar data to teach a Random Forest classifier to predict the safety of crossing the road \cite{Radwan}. While this multi-modal learning approach only uses laser and radar data, it proved advantageous over alternative approaches.\\
In \cite{Baker}, Baker and Yanco utilize a vison-based system to determine the traffic situation. Their solution uses two cameras with a view to the sides to detect incoming traffic and determine the safety of the crossing. The tracing of traffic situations is also performed during the crossing maneuver. Since this work was carried out for assistive robots, which could be installed in wheelchairs, for example, it was designed assuming the crossing would occur at a designated pedestrian crossing.\\
There are also several works regarding the safe crossing of an intersection by autonomous vehicles. While these works are not directly applicable to the problem of crossing the road by a robot, they do provide some insight into the issue of autonomous maneuvering in an urban environment.\\
The use of communication between vehicles, traffic structures, and traffic lights could be used to enable and improve autonomous operation. This approach, with its possibilities and drawbacks, was explored by Torres and Malikopoulos in \cite{CAV}. Moreover, the communication between vehicles and traffic infrastructure could be utilized by robots to allocate the right of way for the crossing. Autonomous cooperative driving was also explored by Lee and Park in \cite{Lee} and Campos et al. in \cite{Campos}.\\
Approaches to autonomous crossing of intersections without vehicle-to-vehicle communication were also explored. In \cite{CMMPHD}, the authors present a general-purpose multi-sensor tracking algorithm.\\
While the aforementioned works primarily focused on sensor selection and data processing techniques, we primarily focused on utilizing the data to make informed crossing decisions. Consequently, we developed a universal control algorithm capable of leveraging sensor data from various sources as long as they provide the necessary information.\\\\
The problem of autonomous road crossing is complex, requiring not only sufficient sensor data and processing but also a robust and reliable control algorithm. Another important aspect is the safety of the robot and its surroundings. Therefore, it is vital to ensure that all necessary safety measures are in place.\\
Henceforth, there is a need for a legal framework to regulate the operation of autonomous agents in urban environments and public spaces. This framework should ensure the safety of the agent and its surroundings while also allowing for the development of new technologies. The responsibilities of the agent and the human operator should be clearly defined.\\
Until such a framework is in place, the deployment of autonomous agents will be limited to controlled environments, such as factories and warehouses.
